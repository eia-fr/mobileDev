
%%++++++++++++++++++++++++++++++++++++++++++++++++++++++++++++++++++++++++++++++
%%  Packages
\usepackage[english]{babel} 			% Langue
\usepackage[utf8]{inputenc}											% Encodage
\usepackage[T1]{fontenc}											  % Requis

\usepackage[pdftex]{graphicx}										% Images 
\usepackage{fancyhdr}												% Spécifier Entête et pieds.
\usepackage[margin=2.5cm]{geometry}

\usepackage{url}													% URL 
\usepackage{hyperref}
\usepackage{verbatim}												% Texte entre 	\begin{verbatim}  \end{verbatim} ne sera pas interprété
\usepackage{listliketab}											% List spéciale, notament comme tableau
\usepackage{longtable}
\usepackage{booktabs}												% Permet de faire des tableau plus avec des traits
% Exemple:
%\begin{tabular}{llr}
%\toprule
%\multicolumn{2}{c}{Item} \\
%\cmidrule(r){1-2}
%Animal & Description & Price (\$) \\
%\midrule
%Gnat  & per gram & 13.65 \\
%      & each     &  0.01 \\
%\bottomrule
%\end{tabular}

\usepackage{color}		
%% \definecolor{orange}{RGB}{255,127,0} http://en.wikibooks.org/wiki/LaTeX/Colors

\usepackage{tabularx}											% Tableau streatching 
\usepackage{colortbl}											% Permet des tableaux exotiquement colorier 
\usepackage{wrapfig}											% Permet d'alligner une figure à guache ou à droite
%\begin{wrapfigure}{r}{40mm}
%  \begin{center}
%    \includegraphics{toucan.eps}
%  \end{center}
%  \caption{The Toucan}
%\end{wrapfigure}
\usepackage{rotating}
\usepackage{amsmath}
\usepackage{subfig}
\usepackage{pdfpages}										% inclus pdf http://www-hep2.fzu.cz/tex/texmf-dist/doc/latex/pdfpages/pdf-ex.pdf
\usepackage[subfigure]{tocloft}
%\begin{figure}[htp]
%  \begin{center}
%    \subfigure[Original image]{\label{fig:edge-a}\includegraphics[scale=0.75]{toucan.eps}}
%    \subfigure[After Laplace edge detection]{\label{fig:edge-b}\includegraphics[scale=0.75]{laplace_toucan.eps}} \\
%    \subfigure[After Sobel edge detection]{\label{fig:edge-c}\includegraphics[scale=0.75]{sobel_toucan.eps}}
%  \end{center}
%  \caption{Various edge detection algorithms}
%  \label{fig:edge}
%\end{figure}
\usepackage[table]{xcolor}								
\usepackage{url}	
%\usepackage{lscape} 								%% Texte en landspcae
%\begin{landscape}
%notre texte
%\end{landscape}

%%++++++++++++++++++++++++++++++++++++++++++++++++++++++++++++++++++++++++++++++
%%  macros
\newcommand{\todo}[1]{\colorbox{red}{\color{white}:TODO:}#1}


